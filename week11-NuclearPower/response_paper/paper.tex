\documentclass[12pt]{turabian-researchpaper}

\usepackage[utf8]{inputenc}
\usepackage{csquotes,ellipsis,xurl}
\usepackage[pass, letterpaper]{geometry}
\usepackage{biblatex-chicago}
\usepackage{hyperref}

\newcommand\question[2]{\noindent\textbf{#1 \quad #2}}

\title{Week 11}
\subtitle{Response Paper}
\author{Eric Altenburg}
\course{HST 415: The Nuclear Era}
\date{\today}

\raggedright


\begin{document}
\maketitle

\question{1}{A question, reaction, etc., to the lecture and/or reading}

	\begin{itemize}
		\item You mention that the UK failed their H-bomb test, did the US ever fail a nuclear bomb test? Seems like they have always succeeded.
		\item In the NPT, each NNWS has “inalienable right” to peaceful nuclear tech, but does that mean a very impoverished country can just ask for one and get it with no funding?
		\item What happens if you didn’t give a 90 day notice for withdrawing from the treaty?
		\item Bertrand Goldschmidt: “Not only did we take the girl when she was a virgin, but we made her pregnant.” An example of the weird sexual phrasing in nuclear talk found by Cohn?
	\end{itemize}

\question{2}{Lehrer wrote his song in 1965. How well has it aged, in your mind?}

	At the beginning of his act he goes over the news of China detonating their own "gadget" and it reminds me of Cohn's article discussing this sort of technostrategic talk. It seems like the normal population was aware of such a weird way of talking nuclear technology, and decided to make fun of it or joke about it, especially in this song.

	As for the song portion, it certainly seems like this has aged well because nuclear weapons are still being talked about today with the threats that they pose; be it to the US or the world. Lehrer portrays the song in a manner such that in regard to the debate, is not not pro nuclear proliferation. It follows the narrative of: if we keep letting nations have these bombs, then it requires that the US have an expansive nuclear intelligence program so we do not get attacked. And when discussing specifically which nations are getting the bombs, it sounds like they are portrayed in such a manner that they all have the sole intention of bombing the US by speaking about our relation with the said country. Even now, there seems to be talk every now and then when North Korea gets a bomb or tests it, and much like the song sings about, I feel like the first thought that pops in the mind of Americans is that they are going to use it on them.

\question{3}{As discussed in the lecture, there is a debate about proliferation that can be summarized as, "more is better" (e.g., the more countries that have nukes, the safer the world will be) versus "less is better" (it would be better to have a world with as few nuclear powers as possible). Which of these, in 2020, do you find more compelling? Would the risk of conflict be lower or higher if, say, South Korea, Japan, and/or Iran had nuclear weapons? Or if France and the United Kingdom did not have them? You are free to editorialize on this, but please listen to the lecture before writing this up. :-)}

	I find the side of "less is better/more is worse" to make more sense with nuclear proliferation. Like it was mentioned in the lecture, I feel as though these two points compliment each other very well. The first being that nuclear weapons are not controlled by states, statesmen, but instead by imperfect, normal human beings inside imperfect, normal organizations. And the second being that, "just because the Cold War didn't turn 'hot' doesn't mean it didn't come close, and future proliferators may not have as much discipline with their nukes as past ones." Specifically with regard to the latter, if we continue with proliferation and allowing all these nations to have their own nuclear weapons programs, how will it be possible to prevent another arms race as it was mentioned that the Cold War was teetering on the edge of all out attack? Even if there was a nuclear agreement passed in which it allowed for expansive monitoring to prevent vast nuclear weapons development, how will this be different from past policies seeking to do something similar? They all tend toward failure as foreign entities are requesting access to highly secretive areas of a nation. Say there was a perfect policy developed, by allowing more and more nations to have access to the technology, the amount of moving parts keeps on building up and like it was previously mentioned, imperfect humans are running these operations, so mistakes are inevitable and in this area mistakes can be extremely deadly.


\vspace*{\fill}
\noindent\textit{I pledge my honor that I have abided by the Stevens Honor System.} -Eric Altenburg

\end{document}