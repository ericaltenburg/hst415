\documentclass[12pt]{turabian-researchpaper}

\usepackage[utf8]{inputenc}
\usepackage{csquotes,ellipsis,xurl}
\usepackage[pass, letterpaper]{geometry}
\usepackage{biblatex-chicago}
\usepackage{hyperref}

\newcommand\question[2]{\noindent\textbf{#1 \quad #2}}

\title{Week 11}
\subtitle{Response Paper}
\author{Eric Altenburg}
\course{HST 415: The Nuclear Era}
\date{\today}

\raggedright


\begin{document}
\maketitle

\question{1}{Please prepare a question or a comment! Something you could bring up in discussion!}

	\begin{itemize}
		\item Was the TRIGA reactor a part of the NPT treaty?
		\item Does the pool water ever get emptied in the BWR?
		\item Why do the cool down towers have that iconic shape of being "slim in the middle but wider at the top and bottom"?
		\item Was there a similar anti-nuclear movement in Russia following Chernobyl or was it squashed by the Russian Government since it kind of downplays their nuclear capabilities?
	\end{itemize}

\question{2}{Characterize the arguments of Perrow (a sociologist who has long written on the complexity of nuclear power plants and how "normal accidents" will occur no matter what), Walker (the former official historian of the Nuclear Regulatory Commission, the government agency that regulates nuclear power), and Aytbaev et al. on nuclear accidents and nuclear risks. All three are making arguments about the "meaning" of these accidents. Which do you find most compelling?}
	
	Charles Perrow's entire argument put forth in his article makes the claim that because systems like nuclear reactors and chemical plants are nonlinear, when there are multiple failures that happen at once, no designer or worker is going to be fully prepared for the event. Simply put, there are far too many permutations of possible failures where there can be a comprehensive guide to each situation. Couple this with the human error, greed, desire for prestige, and politics, then these regulations and warnings put in place become as effective as stopping a bullet with tissue paper. He then builds on this "chaining" of terrible events by saying that the consequences of someone's failures can only be as catastrophic as the organization they work for. He says, "the larger the organization, the greater the concentration of destructive power. The large the organizations, the greater the potential for political power that can influence regulations and ignore warnings" (Perrow 51). Aside from these various problems that lead to the big issues of these complex systems and big organizations, Perrow also discusses what happens after they happen and how they are often coped with. Of course there are some situations where the outcomes are good and first-responders handle the situation, but there are also equally messy and corrupt situations. For Chernobyl, Russia did not admit there was an accident even though Sweden found evidence of such, and with the BP oil spill, everything was handled in such a way that scientists and professionals were not allowed near the scene. Organizations that cope like this do it for a control over the narrative, they need containment, but this only leads to a worse aftermath.

	J. Samuel Walker discusses the counts of the Three Mile Island accident and how the culmination of issues which happened at the reactor that morning led to partial meltdown. Also noted was the movie \textit{The China Syndrome} which was released just two-weeks prior and likely led to the scaled up reaction the accident got. The public viewed the incident as being on par with some of the most devastating attacks around the world during the given time period. This includes events such as the Titanic sinking, the Challenger exploding, and the events that happened on September 11, 2001. Because of the bad press the accident got from the movie, the accident is portrayed in a much harsher light when in reality, it was found that the accident did not cause an immediate loss of life, the effect on public health as was most negligible, and it did not damage any of the surrounding region. 

	Bulat Aytbaev et al. simply explains the stories of what went wrong with the major three nuclear accidents—three mile island, Chernobyl, and Fukushima. However, it seems like the article is geared towards the battle with climate change and how if the demand for energy and power keeps rising, then the current fear toward nuclear energy will lead to a mainly coal driven industry, and this will make situations much worse than they already are.

	Based on these three tellings of the events and what they mean for the future, I feel as though the first by Perrow is the most convincing. The events that lead up to the accidents of nuclear reactors like Chernobyl, three mile island, or even Fukushima are a combination of issues that no designer of the reactor could have predicted. Couple the various gadget failures along side natural disasters like what happened in Japan, and it present new challenges that were not thought to ever possibly happen. And on top of that, the notion of the problems a single worker causes are only as big as the organization I feel is also true. In the case of Chernobyl for example, the few workers who were there that caused the reactor to meltdown were exacerbated due to the governments refusal to admit its errors and take the necessary precautions. 
	

\question{3}{What do you think the future of nuclear energy is for the United States? For the world? What do you think it will be, versus what do you think it ought to be? Is there a difference? If so, why?}

	I feel like the current rush for solar and wind power is going to take the place for green and efficient energy, however, because of this push for cleaner sources, coal and other like industries will likely fall. With this fall of coal, I think nuclear will make a resurgence in popularity though not as much as it was at the height of the nuclear power era. Moreover, due to the population in educated people I feel like this will also lead to more people in favor of nuclear reactors and like energy. Though with this being said, this is what I would hope for the world to come to in the coming years, but with the overhead of costs to establish a reactor, the likelihood of this happening is very slim. 


\vspace*{\fill}
\noindent\textit{I pledge my honor that I have abided by the Stevens Honor System.} -Eric Altenburg

\end{document}