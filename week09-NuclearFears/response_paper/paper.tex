\documentclass[12pt]{turabian-researchpaper}

\usepackage[utf8]{inputenc}
\usepackage{csquotes,ellipsis,xurl}
\usepackage[pass, letterpaper]{geometry}
\usepackage{biblatex-chicago}
\usepackage{hyperref}

\newcommand\question[2]{\noindent\textbf{#1: #2}}

\title{Week 9}
\subtitle{Response Paper}
\author{Eric Altenburg}
\course{HST 415: The Nuclear Era}
\date{\today}

\raggedright


\begin{document}
\maketitle

\question{1}{Come up with a question!}

	\begin{itemize}
		\item Is the psychology behind these abstract systems and the technostrategic language the same as why comedy (specifically referring to the Tom Lehrer songs) works—that is allows us to think and imagine the unthinkable?
		\item Do you find yourself using such a technostrategic language, and if so, what are your experiences with it?
		\item Does technostrategic terminology still apply today or has it shifted more toward an English form?
		\item I noticed in the "Duck and Cover" film that the school it was filmed at was mainly white. At this time, I think there were still some places where it was segregated by ethnicity, were these people given the same instructions/facilities that would protect them from a nuclear detonation as well? 
		\item Building off the previous question, were the places marked with an S also segregated?
	\end{itemize}

\question{2}{What do you make of the Cohn article? What's the argument? How can you apply her analysis to what we've done in this class so far?}
	
	Her research into the language being used to discuss nuclear topics and policies was very interesting. What was surprising to me was the language itself did not allow for the expression of opinions from people who took a personal approach to the topics. Cohn felt restricted in what she could and could not say, and if she spoke traditional English, then those who were in the room would respond to her as if she was ignorant, simpleminded, or even both. And I believe this is where her argument is focused—the language was constructed by those in power who were white men, and as such, it not only required people to speak the same language as them, but also heavily restricted the opposing views of others. This would allow them to guide the topic in one direction without opposition which is why all the policies seem so abstract and out-of-touch with reality. An example that I can connect to is one which we have talked about in class; Kahn's book. He talked about the idea of nuclear warfare and policies as if it were just a numbers game. Specifically when discussing the death count as a result of the war, it seemed very disconnected because while 50 million dead is not good, it is certainly not as bad as 150 million. 

\question{3}{\textit{If you get to them...} what are your thoughts on "Duck and Cover" and "The House in the Middle"? Do you find them compelling as communicative media? (From a factual perspective, they are both \textit{correct for their context}, but that doesn't mean they communicate that well.) Why or why not? What would Cohn say about these (especially "The House in the Middle")?}

	I had actually previously seen "The House in the Middle" from \href{https://www.youtube.com/watch?v=kn0XEa4B144}{this Youtube video} earlier this Summer and I still find it funny how the entire thing seemed like an over-the-top product placement. It seems as though its message was to paint up the old houses and clean up the neighborhoods or else people will pay the price, and coupling this message with the cooperation of a federal agency seemed to give it some validation. At the time of its release, I am sure this was highly convincing marketing where people had to do everything in their power when nuclear warfare was a looming threat, but today, this seems like a blatant marketing ploy targeting the fears of the American people at the time. Had Cohn watched "The House in the Middle" I feel like she would specifically talk about the idea of the film trying to establish a more "up-kept" idea of living in suburbs. This idea of a better kept lawn and house in 1954 was likely the job of a female figure or a maid, and so by this video telling people to be more neat, it is indirectly targeting the female population.  

	For the "Duck and Cover" video, I feel as though it was good in communicating its ideas to the public. Clearly this video's target audience were younger children, and by placing the settings of the film around school grounds and in various situations like in school buses, it really allows for children to better understand what to do in a situation when a nuclear bomb is set off.

\vspace*{\fill}
\noindent\textit{I pledge my honor that I have abided by the Stevens Honor System.} -Eric Altenburg

\end{document}