\documentclass[12pt]{turabian-researchpaper}

\usepackage[utf8]{inputenc}
\usepackage{csquotes,ellipsis,xurl}
\usepackage[pass, letterpaper]{geometry}
\usepackage{biblatex-chicago}
\usepackage{hyperref}

\newcommand\question[2]{\noindent\textbf{#1 \quad #2}}
\renewcommand\part[2]{\vspace{.10in}\textbf{(#1) \quad #2}\par}

\title{Week 12}
\subtitle{Response Paper}
\author{Eric Altenburg}
\course{HST 415: The Nuclear Era}
\date{\today}

\raggedright


\begin{document}
\maketitle

\question{1}{For your response paper, just write up any reflections, thought, questions, etc. that came to mind while watching the lecture and reading the above paper!}

	\part{1}{Questions}
	\begin{itemize}
		\item Although the terrorism injuries and deaths are considered to be low compared to if a nuclear bomb were set off, what would a similar chart look like around the world? Does the US have comparably less deaths/injuries due to terrorism than any other nation?
		\item ... you have copies of Phillips’s paper... does that mean maybe we can read it?
		\item For nuclear briefcases and other similar smaller bombs, what would the point of such a small bomb be if the explosion would also be small as well? It sounds like there is a point of diminishing returns.
		\item Was the reason for the really big laser not igniting because of the drop-off in power?
	\end{itemize}

	\part{2}{Other stuff about how I feel about stuff}

	Prior to this lecture, there would be previous discussions of people throughout the nuclear era—especially early on—when people would go their entire life trying to say how a technology or idea is extremely dangerous. I never understood why people would dedicate their entire lives for efforts like these, but after watching this lecture I think I understand. They choose to go public with ideas about nuclear weapons because they truly believe the world and the human race is in grave danger. They have the proper knowledge and the know-how about the nuclear operations, and so they also know that it is very possible that two nations could result in the collapse of civilization; they are driven mad trying to prevent it. Because of this, it finally makes sense to me why they would go through their entire life preaching what they think is extremely dangerous and should not be pursued. 

	In terms of peaceful nuclear, I saw this video \href{https://www.youtube.com/watch?v=Nh5Tx1QLKBI}{https://www.youtube.com/watch?v=Nh5Tx1QLKBI} where it discusses the idea about small modular reactors (SMRs) and how they may combat the issue with upfront cost that traditional nuclear reactors face. The SMRs themselves are more compact and as a result produce less electricity, but with them being so small and easy to make, it allows for power production to happen a lot sooner. And as the facility is being built, it can house more and more of these SMRs; up to twelve total providing I think more than what is generated in reactors used in Georgia (I am not sure what type of reactor it is being compared to so this may be a moot point). By producing power at such an early stage, I feel as though this can offset some of the total cost that is required to make such a facility and can probably lead to a more nuclear energy focused future.

	As for the article, it is really unsettling how the author lays out the concept for creating a nuclear device in America's backyard. It seems like the cost for making such a weapon seems high, but I feel like money may not be considered to be a factor when it comes to massive projects such as this. All it takes is a private corporation, or a foreign entity to fund the project and money is no longer a concern. 

	The other aspect of the article that interested me was the problem of obtaining the highly enriched uranium. In lecture, when discussing the material unaccounted for, it was stated that around two percent of the material is often lost in the process and nothing can be done about it; it is lost naturally. But if the production is about eight tons of material, then with an error rate of two percent it comes out to one hundred and sixty kilograms of material that has been lost. With this quantity being unaccounted for, a comparatively small amount of material of say six to ten kilgorams that can go "missing" is alarming. All it takes is a connection with an inside man (or woman) at a plant, and over the course of a year, it seems like it is eerily possibly for a terrorist organization to obtain the necessary amount of material they need.

\vspace*{\fill}
\noindent\textit{I pledge my honor that I have abided by the Stevens Honor System.} -Eric Altenburg

\end{document}