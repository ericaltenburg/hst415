\documentclass[12pt]{turabian-researchpaper}

\usepackage[utf8]{inputenc}
\usepackage{csquotes,ellipsis,xurl}
\usepackage[pass, letterpaper]{geometry}
\usepackage{biblatex-chicago}
\usepackage{hyperref}

\newcommand\question[2]{\noindent\textbf{#1 \quad #2}}
\renewcommand\part[2]{\vspace{.10in}\textbf{(#1) \quad #2}\par}

\title{Week 14}
\subtitle{Response Paper}
\author{Eric Altenburg}
\course{HST 415: The Nuclear Era}
\date{\today}

\raggedright


\begin{document}
\maketitle

\question{1}{As usual, generate a question! Something you can — and might! — bring up in discussion! Or two questions!!}

	\begin{itemize}
		\item If a nation were to develop a viable solution to shooting down nuclear missiles, do you think it might be possible to spread this technology to other nations to protect them from other nuclear power nations who might be able to blackmail other countries basically defeating one of the main motives for nations to obtain nuclear weapons?
		\item On the incident of September 1983 with Stanislav Petrov, if he went through the motions and reported to the chain of command, would it have started the "dead hand" system?
		\item Was the Nuclear Winter argument used by the anti-nuclear movements?
		\item During the Reyjkavik summit there was the topic of the US and the USSR completely getting rid of their nukes. Aside from these two nations removing their deposit
		\item How accurate do you think the command and control was in \textit{The 2020 Commission Report on the North Korean Nuclear Attacks Against the United States}?
	\end{itemize}


\question{2}{What's your general take of the 2020 Report? Give me a brief review. I have read it, so you don't need to summarize it. (How does it feel to be reading it in 2020?)}

	I am not sure if it is because I do not read too much realistic fiction, but the plot to this book seemed very surreal and eerier to me. Reading that all of these events were happening in the present day with the current standing government officials and their specific mannerisms made me genuinely believe that the events that led to the nuclear bombings on the US would possibly be done. To me, it seems like the whole event happened because of a lack of communication. With North Korea's main network down coupled with being in a cramped bunker means it was a breeding ground for irrational theories to be amplified.

	The moment in the book that made me seriously think about what was happening was South Korea's decision to launch their own missiles at North Korea in response to the shootdown of BX 411. It caught me off guard when they had to consider whether to run their response past the United States as if it was their parental figure. I did not realize that with the policies and formalities, the autonomy of a nation to make their own decisions seemed to be limited. This thought sparked a mini conflict within me about whether I would have consulted the US if I were in President Moon's shoes. Ultimately I came to the conclusion that I would have gone ahead with the launches anyway simply because it is my nation, and killing my people—102 of which were \textit{children}—is absolutely unacceptable; however, I would likely not attack Kim Jong-Un's palace and only the air base since they were the ones who shot the flight down. It can be said that talking with the US would allow me to mull things over with time, but seeing as though I was in the bunker at the Blue House first for a few minutes with no one but myself and my thoughts, my mind would be racing and my idea to follow-up with an attack would only be reinforced. Every second that went by where I would be playing the diplomacy card with the US was a second not doing the families of the children justice.


\question{3}{ Finally, take up the question I ask at the end of the lecture. Do we live in a third nuclear age? Or have we just never left the first? What's your sense of this, having taken this class?}

	I believe splitting up the time period from the inception of nuclear energy up to present-day is a good way of categorizing what has taken place during this relatively expansive time period. With that said though, this does not mean they are separated into their own distinct and separate eras. I feel as though what constitutes an era is when there is a monumental shift or definitive end in the topic. Because these nuclear weapon topics and technologies tend to bleed into each other, there is no clearly defined end to a certain topic that would mark the end of a nuclear era with the beginning of another. 

	I would argue that we never actually left the first nuclear era which started with the discovery of splitting uranium atoms. Nuclear energy seem to have been one continuous strand of findings that built upon one another. We would not have atomic bombs if someone did not think to use the splitting of atoms to be the energy delivery system; we would not have the Teller-Ulam design if it were not for the atomic bomb to be the primary igniter. These are all situations in which a new advancement has come about due to old findings. There were no monumental shifts or ends to the topic of nuclear energy, only a steady forward stream of progress has come about over these many decades.

	

\question{4}{LASTLY... and this is just for my own use... are there things we didn't get to spend time on in this class that you wished we had spent more time on? The answer can be "no, the class is perfect," but that's less useful to me that a real answer!}

	Let me just say, the time I spent in discussion and learning about the nuclear era is by far the most memorable and enjoyable academic experience I have had the pleasure of experiencing here at Stevens. This was the only time I was able to branch out and take a class I was genuinely interested in. I liked the wide range of topics covered—from the inception of the bomb, to the nitty-gritty details of the forces that are involved with splitting the uranium atoms, there was never a moment where I thought the material was not interesting/engaging. 

	The only suggestion I have for the course so far is to potentially change the layout of due dates on the weeks where there are short papers due. The four short papers are often due around the same time as other assignments from classes that also have $\sim 4$ assignments given throughout the semester. So with the short paper and other class assignments due in the same week, adding the weekly response papers I think adds a bit more on the plate. Maybe scrapping that particular weekly response paper and instead, have the next week's response questions mixed in with some of the skipped week's questions as it might give students more time during the short paper due dates. In my own experience, around these due dates I often would have to set lectures to 1.5x speed and skim through the response paper sources, which was slightly upsetting as I enjoyed the response paper topics but could not give them the time they deserved.

	But nonetheless, thank you for this course :)


\vspace*{\fill}
\noindent\textit{I pledge my honor that I have abided by the Stevens Honor System.} -Eric Altenburg

\end{document}