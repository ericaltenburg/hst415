\documentclass[12pt]{turabian-researchpaper}

\usepackage[utf8]{inputenc}
\usepackage{csquotes,ellipsis,xurl}
\usepackage[pass, letterpaper]{geometry}
\usepackage{biblatex-chicago}
\usepackage{hyperref}

\newcommand\question[2]{\noindent\textbf{#1 \quad #2}}
\renewcommand\part[2]{\vspace{.10in}\textbf{(#1) \quad #2}\par}

\title{Week 13}
\subtitle{Response Paper}
\author{Eric Altenburg}
\course{HST 415: The Nuclear Era}
\date{\today}

\raggedright


\begin{document}
\maketitle

\question{1}{Please generate a question!}

	\begin{itemize}
		\item This is approaching War Games territory, but I if you believe that the launch officers will not say no, then should we just cut them out and replace them with a computer? As in give the previous level (C3) a direct line to the nukes, not as in “implement AI like in the movie.”
		\item Has it ever been debated about making nukes more rugged so if they are being launched and a country is trying to shoot it down, it will be strong enough to withstand the damage? Or is the counter of this to always err on the side of caution?
		\item Has there been any effort for the nations to meet and come up with worldwide standard for who decides and how nuclear weapons are used (as in C2/C3, not the international talks done after WWII)?
		\item Out of all the military people (i.e. launch officers) who you’ve talked to who said they would not say no to a nuclear launch, was this in reference to a retaliative or preemptive strike?
	\end{itemize}


\question{2}{How do \textit{you} think nuclear launch authority ought to be arranged in the United States? Explain your reasoning and why!}

	I think my preferred system with how the nuclear launch authority ought to be arranged in the United States is a combination of multiple different policies put into place in different countries.

	The first country's approach I liked was Russia. Their three person system to make a decision based on majority seems like a viable solution for keeping the President in check when they may make an irrational decision or are under extreme situations which can cloud their judgment. While the probability that the President is having an off day is likely very slim as it is, it does not hurt to further reduce this probability, and to do so you can add two others to the mix which would require at least two people to be having an off-day.

	The other country's approach I liked was the United Kingdom. Their statement that the bomb is a purely political decision I feel is true, and for the United States to partially put it in the hands of the military, a non-political branch of the government, it simply does not make sense. Instead, but placing it purely in the hands of those who are adequately engaged in international and domestic politics, it would allow for a more democratic approach since the people of the United States elect those who would be saying whether to use the bombs or not. Additionally, placing the bomb in control of the civilians would not allow for the military to abuse loop holes and interfere with the nuclear launch process as it was discussed in lecture.

	In its current state, I believe that one person who can truly veto the decision to launch a nuclear bomb rests in the military is flawed. The missile men who are in charge of launching the nuclear bombs are hand picked and trained to essentially not say no to a launch order. It was seen in the podcast where if someone shows a shred of doubt in the launch process, then they are typically barred from any sort of military career which severely reduces the possibility of anyone who could possibly say no to launching. Therefore, I feel it should be possible for anyone involved in the process of nuclear launching to simply say "no."


\question{3}{What's your reaction to the podcast and the Always/Never clip?}

	For the podcast, I found it a bit surprising at first that they barred Harold from ever having a career in the air force just because he asked a seemingly reasonable question. If there were extensive checks and balances placed on the missile men, then surely there should be just as many coming from the origin of the command. However, it seems that the military is constructed in such a way where it fully endorses the use of nuclear weapons; they simply want a yes man. Because of this, it begs me to ask the question of: why bother having missile men if you just want someone to do as they are told? It would make sense for them to be replaced entirely so that there is no possible "weak link" in the chain of the nuclear launching process.

	In regard to the video, I still think it is outrageous that the creators of the Chrome Dome operation did not consider the failure rate of the bomber planes per mile flown vs the total amount of miles they wanted them to fly. Surely they must have known that mathematically this was a very risky operation for the America's people and their allies. And to think that a catastrophe was stopped because of \textit{one} fail-safe is almost comedic; a chunk of the United States was saved by \textit{one} singular switch. While it is great that the bomb never went off, I think I fall on the side of "this should have never happened to begin with." To me, the whole operation was doomed to fail from the beginning, but officials were too stressed and caught up with the cold war to truly understand the potential consequences of their actions. It reminds me of the military's attitude when developing the atomic bomb and the dumping of waste, they were living in the present moment while leaving any potential issues for the future—if there was one.


\vspace*{\fill}
\noindent\textit{I pledge my honor that I have abided by the Stevens Honor System.} -Eric Altenburg

\end{document}