\documentclass[12pt]{turabian-researchpaper}

\usepackage[utf8]{inputenc}
\usepackage{csquotes,ellipsis,xurl}
\usepackage[pass, letterpaper]{geometry}
\usepackage{biblatex-chicago}

\newcommand\question[2]{\noindent\textbf{#1: #2}}

\title{Week 8}
\subtitle{Response Paper}
\author{Eric Altenburg}
\course{HST 415: The Nuclear Era}
\date{\today}

\raggedright


\begin{document}
\maketitle

\question{1}{A question!}

	\begin{itemize}
		\item The idea that Kahn puts forward—executive staff choosing a posture and then telling the other departments for it to trickle down further to let the appropriate people debate—does this not take up even more time than it already does? I feel like this will take too long and by this point, the enemy can retaliate or make a move.
	\end{itemize}

\question{2}{What, ultimately, is Kahn's argument about how we should think about thermonuclear war?}
	
	Khan believed with the power these nuclear weapons held and without proper planning or policies, they would surely lead to a catastrophe causing incredible amounts of damage to the world in just a few years; even before the year 2000. In general, what Kahn sought to achieve with his book is to help anticipate, avoid, and alleviate such crises. More specifically, what Khan talked about was a proper deterrent. That the United States must present itself in such a way so a retaliation strike after the Soviets attack would be far too risky for them. And no matter how advanced the Soviets feel their tactics are or their fire power, if the United States can show that a counter attack will be devastating, then the Soviets will rethink their strategy as it would involve massive casualties and destruction on their behalf; the risk is too great.

\question{3}{What is your reaction to Kahn? In his time (as the Newman piece makes clear) there were people who found his approach repulsive and needed to say so in uncertain terms. Do you relate to that, or not? Probe your response to him, and your feelings about the Newman review as well.}

	I found my reaction while reading the methods and policies Kahn proposed to line up with what Newman said. It seemed to me that aside from the topics he discussed in detail, the overall feel I got from the article was that nuclear warfare was almost unavoidable, and this carried with me throughout the reading. The way this carried with me while reading the article is perfectly summed up by Newman where it was said, "this is a moral tract on mass murder: how to plan it, how to commit it, how to get away with it, how to justify it" (Newman 1). His book gave off this idea of him being inhumane in his view of the death counts and such, and I think it shows in one of the tables he used where it depicts the amount dead from a nuclear war and how long it would take for an economic recuperation. To me it just seems like this was a numbers game to him, it completely dehumanized the topic making it feel as though it had less overall weight when in reality, what was being discussed was hundreds of millions of people being dead.

\question{4}{What do the Lehrer songs — which were well-received by the liberal literati of the 1960s — tell us about how the missile age was understood by that demographic? What do you make of his use of humor in addressing such dark topics? (Something which becomes very prevalent by the 1960s.) What kind of anxieties are hiding beneath the laughs?}
	
	I find it interesting how this idea of coping with such dark topics has generally stayed the same throughout the years, even now, although we do not necessarily sing it as it was in the videos, we certainly show the same attitude in terms of our comedic acts and especially in memes. The purpose of such an approach I feel is to mainly introduce the topic to a different audience and make it more approachable, and as such it reaches more audiences. In this case, while discussing the onset of World War 3, putting this in a sort of comedic song form hides the dark nature of what Tom Lehrer is talking about. When the solder is flying above getting ready to drop the bomb, it shows that the idea of missiles goes both ways. The mother is hiding in her own bunker waiting for maybe another mother's son to drop a similar bomb on each other. 

	The reaction the audience gives from hearing these scenarios about WWIII would make it seem as though they do not care about the topic and merely see it as a laughing point, but when he mentions that the war can end in a few hours it is clear the audience agrees. They know it is true, and I believe that is because of the capabilities missiles had. Up until then, the Americans saw that dropping the atomic bombs on Japan in WWII led to a victory and end to the war—although it was not the only reason—and to them, by using the same ideology with missiles that can travel farther distances and potentially end wars without having to step foot on the ground, it makes it seem as though they have powerful war-ending capabilities. 

	The anxieties behind each wave of laughter from the jokes told about Wernher von Braun and WWIII really show how they know what they are hearing is true, yet in the moment the context that it is provided in encourages laughter. It almost makes the topic seem smaller than it actually is—it is more approachable—and as a response since the audience knows the weight each of these two points have, it is simply funny to them.


\vspace*{\fill}
\noindent\textit{I pledge my honor that I have abided by the Stevens Honor System.} -Eric Altenburg

\end{document}