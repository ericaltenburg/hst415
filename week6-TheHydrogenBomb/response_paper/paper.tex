\documentclass[12pt]{turabian-researchpaper}

\usepackage[utf8]{inputenc}
\usepackage{csquotes,ellipsis,xurl}
\usepackage[pass, letterpaper]{geometry}
\usepackage{biblatex-chicago}

\newcommand\question[2]{\noindent\textbf{#1: #2}}

\title{Week 6}
\subtitle{Response Paper}
\author{Eric Altenburg}
\course{HST 415: The Nuclear Era}
\date{\today}

\raggedright

\begin{document}
	\maketitle

	\question{1}{Questions about this that came to mind during lecture or the readings — at least one good one!}

		\begin{itemize}
			\item In the lecture video, you mapped the 10 MT bomb as a surface detonation, why is that? Does the H-bomb require a surface detonation as opposed to an air burst like the Japan A-bombs?
			\item What percentage of the material in the Teller-Ulam design is actually fissioned compared to the Nagasaki? If I recall correctly, percentage wise, the amount fissioned is relatively low.
			\item Does the moral argument go out the window now since if they can make it first, then since they think they're more moral than the enemy, they can use it for good?
			\item jfkdls
		\end{itemize}

	\question{2}{What do you make of the GAC's argument? Do you think you would have found it persuasive in 1949? Do you find it persuasive today, knowing what happened later?}

	\question{3}{What is York's main argument, and what do you think of it?}

	\question{4}{Lastly — in general (and you can use the extra primary sources for this, or not), where do you think you would have landed on the H-bomb debate if you were a scientist with classified information in 1950? How do you think it would change if you were just a person without access to any classified nuclear weapons information? Do you think Truman was right in ordering the crash program, ultimately?}
		

\vspace*{\fill}
\noindent\textit{I pledge my honor that I have abided by the Stevens Honor System.} -Eric Altenburg

\end{document}