\documentclass[12pt]{turabian-researchpaper}

\usepackage[utf8]{inputenc}
\usepackage{csquotes,ellipsis,xurl}
\usepackage[pass, letterpaper]{geometry}
\usepackage{biblatex-chicago}

\newcommand\question[2]{\noindent\textbf{#1: #2}}

\title{Week 6}
\subtitle{Response Paper}
\author{Eric Altenburg}
\course{HST 415: The Nuclear Era}
\date{\today}

\raggedright

\begin{document}
	\maketitle

	\question{1}{Questions about this that came to mind during lecture or the readings — at least one good one!}

		\begin{itemize}
			\item In the lecture video, you mapped the 10 MT bomb as a surface detonation, why is that? Does the H-bomb require a surface detonation as opposed to an air burst like the Japan A-bombs?
			\item What percentage of the material in the Teller-Ulam design is actually fissioned compared to the Nagasaki? If I recall correctly, percentage wise, the amount fissioned is relatively low.
			\item Does the moral argument for not producing the bomb go out the window since the US can make the Super first and because they will inherently they think they're more moral than the enemy, they can use it for good?
			\item In the GAC reports, it is said that, "Mankind would be far better off not to have a demonstrations ... until the present climate of the world opinion changes." What about the current climate is making them not test? The outcomes of Hiroshima \& Nagasaki?
			\item In the event that the USSR uses a Super bomb on us, it is said that reprisals by our large stock of atomic bombs would be comparably effective to the use of a Super. How many bombs did we have precisely?
			\item It seems that one of the major reasons for the US not making the Super is because of moral reasons. However, had this crash project idea been put forward before or replaced the A-bomb project, would they still be arguing for moral reasons? I think it is evident that a little race from another nation is all that was needed to start production, so what if someone said during WWII that the Germans were working on it?
			\item York mentions that when the Soviets tested their bomb on November 23, 1955, a remark from Secretary Khrushchev is said to prove that the design was modeled after the Teller-Ulam design. How did they know about the design? Were there more spies?
			\item What was secrecy like for this project? Was it just as secure as the Manhattan Project or less so?
		\end{itemize}

	\question{2}{What do you make of the GAC's argument? Do you think you would have found it persuasive in 1949? Do you find it persuasive today, knowing what happened later?}

	The GAC make a convincing argument for the time period it was written in, however, in present day it falls short. In 1949, I would have found it persuasive because of the way it portrays the various unknowns about the plan. I also would have agreed with the moral aspect that is talked about because coming off the heels of the Hiroshima and Nagasaki bombings in 1945, I would not want to make something even more devastating; it just becomes overkill. In this regard, I find the language being used to emphasize this funny, they say things like, "the policy of \textit{exterminating} civilian populations," or "its use would involve a decision to \textit{slaughter} a vast number of civilians." However, in present day, my opinion would be different and I would likely not support the GAC's advice. Going back to the moral argument in present day, knowing that the Soviets would still go on to produce a Super bomb, even though we do have a "large stockpile of atomic bombs" to retaliate, I would personally feel more comfortable knowing that the enemy does not have the strongest weapon ever to be created.

	\question{3}{What is York's main argument, and what do you think of it?}

	York aims to shed light on the different scenarios that could have happened if the United States chose to forgo the Super crash program. These situations aims to show that nothing disastrous would have come of them if the US followed the Oppenheimer-Lilienthal advice. For both of the alternate timelines, the only changes to occur are the detonation of the first Soviet Russian Super bomb; one in 3-4 years after the actual date and the other on the same real-world date. Based on the alternate timelines given, I can agree with the notion of saying it would not be the worst thing in the world if we adopted the Oppenheimer-Liliemthal advice. In the very worst-case, the Soviets would have had a Super bomb around the same time as the detonation of Mike and/or Bravo, so it is not as if there would have been a severe nuclear power gap. And in the the best-case, although the Soviets still acquire a Super bomb, they do not do so until several years after the US detonates their own. While there are likely hundreds of thousands of possible timelines of events unfolding at various moments in time, the ones outlined by York seem to be reasonable and in both cases, there is no severe disruption of nuclear armament between the two nations.

	\question{4}{Lastly — in general (and you can use the extra primary sources for this, or not), where do you think you would have landed on the H-bomb debate if you were a scientist with classified information in 1950? How do you think it would change if you were just a person without access to any classified nuclear weapons information? Do you think Truman was right in ordering the crash program, ultimately?}

	If I were a scientist in 1950 who had classified information about the H-bomb, I would likely be pro H-bomb crash course. The reasoning behind my decision lies in the idea discussed in the lecture where "if we technically know how to make it, we're obligated to build it." If we knew how to make the bomb, then purposely ignoring the opportunity to build it for ourselves seems redundant. This leaves room for a bad figure to potentially make it before us, and so we are then morally obligated to create the device before the party with a seemingly more compromised morality. 

	On the contrary, if I were someone who does not have access to classified information, then I would likely think otherwise about the H-bomb crash program. As it was mentioned before, after reading \textit{Hiroshima} by John Hersey, I would be strongly opposed to creating another weapon which can do 100 times more damage. To combat this, I would strongly advocate for opening up international nuclear talks once again about the subject in the hope that other nations would also join.

	Ultimately though, I feel as though Truman was correct in his decision to order the crash program. It was briefly mentioned at the end of the York debate, but only reading the GAC's argument provides contexts in the eyes of a scientist, however, there are political nuances as well. The tensions were rising with the US and the USSR so by taking the back seat here and only producing fissionable material, it would be seen almost as a step backwards in cowardice giving way for the USSR to now become the nuclear super power of the world which is not the nicest option; having Stalin as the king of nuclear is not appealing to me.

\vspace*{\fill}
\noindent\textit{I pledge my honor that I have abided by the Stevens Honor System.} -Eric Altenburg

\end{document}