% Turabian Formatting for Research Papers Template, 2018/08/06
%
% Developed using the turabian-formatting package (2018/08/01), available through CTAN: http://www.ctan.org/pkg/turabian-formatting
%
% Additional document class formatting options:
%
% raggedright: ragged right formatting without hyphenations
% authordate: support for the author-date citation style
% endnotes: support for endnotes



\documentclass[12pt]{turabian-researchpaper}


\usepackage[utf8]{inputenc}
\usepackage{csquotes, ellipsis,xurl}

% Specify paper size with geometry package
\usepackage[pass, letterpaper]{geometry}

% For citations, use the biblatex-chicago package
\usepackage{biblatex-chicago}
\addbibresource{works-cited.bib}

% Information for title page
\title{The Nuclear Power Seesaw}
% \subtitle{A Template based on Turabian's \emph{A Manual for Writers}, 9th edition}
\author{Eric Altenburg}
\course{HST 415: The Nuclear Era}
\date{\today}

\raggedright


\begin{document}

\maketitle

A few years after the atomic bombs were detonated over Hiroshima and Nagasaki, Japan, Edward Teller and Stanislaw Ulam refined an idea to show that making a hydrogen bomb was feasible. From this, it was recommended that a “crash” program be created—similar in how the Manhattan Project was made to develop the first atomic bomb—however, the idea was met with different views as some were for it while others not so much. Those who were for the program consisted of members in the Joint Committee on Atomic Energy (JCAE) and their reasoning was further expanded on in detail during their executive session titled “Development of a Super Weapon.” Here, they analyzed the arguments made by the General Advisory Committee (GAC) to not go forward with a crash program, noting how they might be overstepping their duties and finding general disagreements with their notions. They then reviewed a letter submitted to the President of the United States written by the Chairman, Senator Brien McMahon where he summarized much of the JCAE’s talking points of the hydrogen bomb debate. By the end of the meeting it was established that the members of the JCAE were for the hydrogen bomb crash program due to the valid moral arguments concerning the lives of the American people, along with finding several points of disagreement with the arguments the GAC made.
	
Starting off in the JCAE session, the arguments put forth by the GAC to defend their position in the hydrogen bomb debate were examined closely. A few of the most notable ones include the time estimate the GAC gave for the Soviet Union in regard to how many years it would take for them to obtain a hydrogen bomb. They estimated the US had a decade until the Soviets would obtain one, but Senator McMahon discussed the matter with Dr. Bradbury and Dr. Lawrence to which they came to the conclusion of a much shorter two to three years (Joint Committee on Atomic Energy 1950, 17). This time frame lines up nicely as the current state of the US at the time is beginning to develop Red Fears—where multiple committees were formed to sniff out Communists under the McCarthy conspiracy theory which led to the tragic Rosenberg trial (Wellerstein 2020). Due to the predicted short time period between this crash program debate up to the moment the Soviets may have their first hydrogen bomb, it is reasonable to conclude this upcoming fear of Communism certainly had an influence on a majority of the JCAE in which they feared the Soviets and their predicted time line. Aside from the general disagreement of the decision to not initiate a crash program, the JCAE also took issue with the other proposed decisions by the minority in the GAC report. A tactic was discussed where the US would disclose to the public its decision to not endorse the production of a hydrogen bomb while giving slight details about how to make the said device. This was met with criticism as even if the US were to say this, why would any other nation believe it? The Manhattan Project was one the biggest kept secrets up until the use of the atomic bomb, who is to say that the US would not apply the same strategy for the hydrogen bomb. 
	
After reviewing the report, it was questioned whether or not the GAC were overstepping their responsibilities as a committee. One of their talking points for being against a crash program was the moral issues that could arise with the use of a hydrogen bomb; saying that its use would be the equivalent to “exterminating civilian populations,” or “its use would involve a decision to slaughter a vast number of civilians” (AtomicArchive 1949). With the use of a strong selection of words, it is clear that they believed the use of a hydrogen bomb is simply too powerful to carry out military missions, especially given the fact that the power it can unleash was thought to be unlimited. Upon hearing this, Senator Hickenlooper of the JCAE states, “[the GAC] have indulged in a discussion of morals and not the thing they are set up to discuss” (Joint Committee on Atomic Energy 1950, 30). To further clarify this, he cites the law detailing the exact purpose the GAC, it reads, “there shall be a General Advisory Committee to advise the Commission on the scientific and technical matters relating to the materials, production and research” (33). Using this, it shows how little the JCAE is concerned with the moral argument put forth by the GAC, so much so that they believe they are overstepping their roles and providing advice in an area that does not concern them. Instead, they decide the job of morality is to be determined by the President, defensive forces, or even the Congress; not a committee established to talk purely in terms of scientific matters.

Later in the document, the letter composed by Senator McMahon is read which details his reasons for wanting the US to proceed with a crash program. His overall goal for the letter centers around the idea of the Soviets obtaining a hydrogen bomb first, he says, “if we let Russia get the super first, catastrophe becomes all but certain whereas, if we get it first, there exists a chance of saving ourselves” (41). By starting a crash program immediately instead of delaying, it would give the US a far greater advantage, one where we could have a “bargaining chip” and can leverage our power over the Soviets to prevent an incoming nuclear war. One interesting point he talks about directly defies the argument of the GAC where they believed the job of a hydrogen bomb is simply too powerful for any practical militaristic use. He says that the Soviets have made changes in the layouts and spacing of their air bases and factories noting that they are more isolated and spaced out from one another. This poses an issue as if atomic bombs were to be used, destroying these structures would not be possible by themselves due to the comparatively small blast radius. Rather, a hydrogen bomb should be used in situations like these in order to fully destroy the intended targets with just one device instead of dropping multiple atomic bombs which can be more costly in terms of resources.

Following the detonation of the most powerful device on Japan, the GAC were among those who had strong criticisms regarding an even more powerful weapon. Rightfully so, they were concerned with the morality of its usage along with the sheer manpower and economics involved with another nuclear project; however, on the other side of the argument lies the JCAE. In a meeting analyzing the GAC's reasoning for not backing a crash program, they felt that it is not the place of the GAC to offer moral advice or even offer the arguments they did. They felt as though by allowing the US to take a backseat and letting the Soviets develop a hydrogen bomb, the US would be put in danger; the US would not have a bargaining chip to keep the Soviets in check and prevent them from using such a devastating weapon. Keeping the nuclear power between the nations equal and not one-sided was key in this time period, had it been tipped in favor of the Soviets, then the landscape of the world may look vastly different than how it does now.

\newpage

\begin{thebibliography}{3}
	\bibitem{Three} AtomicArchive. 1949. "General Advisory Committee's Majority and Minority Reports on Building the H-Bomb." October 20, 1949. \url{https://www.atomicarchive.com/resources/documents/hydrogen/gac-report.html}
	\bibitem{One} Transcript of an Executive Session. "Development of a Super Weapon." Joint Committee on Atomic Energy (9 January 1950).
	\bibitem{Two} Wellerstein, Alex. "Nuclear Secrecy, Security, and the Tragedy of Oppenheimer." HST 415: The Nuclear Era (class lecture, Stevens Institute of Technology, October 14, 2020).
	% \bibitem{two} Alperovitz, Gar and Sherwin, Martin J. 2020. "Op-Ed: U.S. leaders knew we didn’t have to drop atomic bombs on Japan to win the war. We did it anyway." \textit{Los Angeles Times}, August 5, 2020. 
	% \url{https://latimes.com/opinion/story/2020-08-05/hiroshima-anniversary-japan-atomic-bombs}.
	% \bibitem{five} Davies, Dave. 2020. "'Fallout' Tells The Story Of The Journalist Who Exposed The 'Hiroshima Cover-Up.'" \textit{Georgia Public Broadcasting}, August 19, 2020. \url{https://www.gpb.org/news/2020/08/19/fallout-tells-the-story-of-the-journalist-who-exposed-the-hiroshima-cover}.
	% \bibitem{three} Fedman, David and Karacas, Cary. 2020. "Opinion: 75 Years On, Remember Hiroshima And Nagasaki. But Remember Toyama Too." \textit{NPR}, August 1, 2020. \url{https://www.npr.org/2020/08/01/896627359/opinion-75-years-on-remember-hiroshima-and-nagasaki-but-remember-toyama-too}.
	% \bibitem{four} Hersey, John. 1946. \textit{Hiroshima}. New York: A.A. Knopf.
	% \bibitem{one} Walker, J. Samuel. "Recent Literature on Truman's Atomic Bomb Decision: A Search for Middle Ground." \textit{Diplomatic History} 29, no. 2 (April 2005): 311-334.
	% \bibitem{six} Wellerstein, Alex. "Using the Bomb." HST 415: The Nuclear Era (class lecture, Stevens Institute of Technology, September 15, 2020).
\end{thebibliography}

\end{document}