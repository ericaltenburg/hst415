% Turabian Formatting for Research Papers Template, 2018/08/06
%
% Developed using the turabian-formatting package (2018/08/01), available through CTAN: http://www.ctan.org/pkg/turabian-formatting
%
% Additional document class formatting options:
%
% raggedright: ragged right formatting without hyphenations
% authordate: support for the author-date citation style
% endnotes: support for endnotes



\documentclass[12pt]{turabian-researchpaper}


\usepackage[utf8]{inputenc}
\usepackage{csquotes, ellipsis,xurl,tocloft}

% Specify paper size with geometry package
\usepackage[pass, letterpaper]{geometry}

% For citations, use the biblatex-chicago package
\usepackage{biblatex-chicago}
\addbibresource{works-cited.bib}

% Numbering the sections
\setcounter{secnumdepth}{2}
\setcounter{tocdepth}{2}
\renewcommand{\cftsecleader}{\cftdotfill{\cftdotsep}}

% Information for title page
\title{The Missing Appendix}
\subtitle{How to Prevent and React to Nuclear Launches}
\author{Eric Altenburg}
\course{HST 415: The Nuclear Era}
\date{\today}

\raggedright


\begin{document}

\maketitle

\tableofcontents

\newpage

The events that unfolded surrounding the shoot down of South Korean flight BX 411 on March 21, 2020 was nothing short of a fault in many channels within several countries. The lack of a strong communication infrastructure on behalf of the North Koreans, preemptive irrational decisions from President Moon of South Korea, and the absence of a proper nuclear missile defense program—which has been on a steady decline for several years in the United States—are all issues that could have been avoided. The near two million lives lost over the course of four days was the result of several inconceivable issues all happening at once. To ensure that like events do not arise again, adequate precautions must be enforced now to give light to a better future.


\section{Retaliate with a proportionate amount of force}

\subsection{Background}
	
	The United States had previously used a war tactic called SCATHE JIGSAW during the Cold War against the Soviet Union where US pilots would fly straight toward Soviet airspace forcing the them to go on alert turning on all radars and readying aircrafts; however, at the last second the US pilots would turn around. This was seen as a successful psychological operation unnerving the Soviets as they would not know what the US bombers were doing flying straight toward them. 

	This same tactic was used on the North Koreans to achieve a similar goal. US bombers would fly toward North Korean airspace causing the military to be on alert, and at the last second, the pilots would turn around. However, as the South Korean flight BX 411 lost its communications equipment for a short while, they accidentally began to fly the same flight path as the US bombers. Because they were unable to communicate with airport towers and they did not know they were flying the bomber path, they did not turn around at the last second like a typical US bomber pilot would, and North Korea mistook it for such a bomber shooting it down.

	President Moon of South Korea got news of the deaths of everyone on board flight BX 411—102 of which were children—and decided to take immediate action without consulting allies such as the United States. He decided to not only attack the North Korean Air Force headquarters located in Chunghwa, but additionally the Kim family compound in Pyongyang. The former was deemed a proportionate response by Admiral Un (Lewis 41), however, the latter was over the top.

\subsection{Recommendation}
	
	Nations must have checks and balances to ensure that leaders like President Moon exhaust all possible options of retaliation while convening with their cabinet before making a decision. The goal of such a discussion is to prevent actions that mimic the latter decision made my President Moon to attack the Kim family compound; this was seen as a direct attack and threat to the Kim family and regime. By only targeting the Air Force headquarters, this action can be interpreted as an "eye-for-an-eye" since this was where the command to shoot down BX 411 came from. But adding the Kim compound to the list of targets was an excessive use of force. 

	If the initiating attack is not something that can be categorized as a wartime attack (i.e. thousands to hundreds of thousands dead), then the response should be proportionate to how it is. Attacking the family compound of a leader can easily be interpreted as a war initiator if the country's main communications are down, as it was in North Korea. Therefore, Kim Jong-un's decision to fire back at the US, South Korea, and Japan seemed like a reasonable plan of action; he believed the country was being attacked on a much larger scale than intended. And thus, if President Moon only attacked the Air Force headquarters, the response would likely not have escalated to extent it did.

% \section{Ensure communication can withstand devastating attacks}
% 	\subsection{background}

% 	- the attacks on NK took out the infrastructure of the nation
% 	- Kim Jon un was not abel to communicate with the outside world, and him being paranoid enough made all decision under extreme pressure with heavy bias.
% 		- it was under the assumption that the US was behind it all when it was in face the SK
% 	- If they knew that the US was not behind the launch, then things would not have escalated

% 	\subsection{recommendation}
% 	- Make sure all countries have a solid means of communicating within the nation and throughrou the world to prevent such disasters from happening. 
% 	- if NK just knew that the US was not behind the attacks, then this would nto have happened

\section{Develop proper weapons defense programs in multiple locations}

	\subsection{Background}

	After an effort by the US to try and destroy as many North Korean launch pads capable of launching nuclear missiles that can reach mainland US, there still ended up being a dozen missiles in the sky. The only way to intercept these missiles is if a set of defense mechanisms in Alaska is able to take them down. However, the systems put in place had several errors and resulted in a low probability of successfully intercepting them.

	This low percentage has been known by US officials for quite some time now, and to offset this, instead of developing proper research into the defense field, officials opted to deploy more faulty machinery with the hopes of compounding these low probabilities to form a better one. While this seems computationally sound, beyond the numbers, there are issues with this line of thinking. According to one critic, "it assumes that the failure modes of the interceptors are independent of one another. But, in practice, if one interceptor fails because of a design flaw, say, it's much more likely that others will do so too for the same reason" (224). The systems deployed had several issues with them. Most notably is that a large percentage had old faulty circuit boards that caused some machinery to malfunction. While it might seem right to replace all the circuit boards, only those affected were replaced leaving some with outdated equipment primed for failure. Additionally, the missiles heading toward the US were equipped with decoys to fool the interceptors further reducing the probability of a successful take down.

	\subsection{Recommendation}

	Those nations that are united under the Non-Proliferation of Nuclear Weapons Treaty should pool resources to research and develop a strategy to better handle nuclear missiles launched from rogue nations like North Korea. 

	Previously, there have been issues with nations such as the US planning to develop a defense system capable of destroying missiles much to the likes of a sci-fi movie, however, this was seen as controversial and caused tensions between the US and Russia. The promise to share this technology to other nations once fully developed did not carry much weight to it, and as a result, Russia felt as though it would be used against them; in other words, the US would be untouchable. 

	Instead, if the nations pool their resources for means of creating a more successful and easily adoptable anti-missile system, then it should be done through a third-party and not be housed within any one specific nation. This would help deter the potential issues similar to the one between US and Russia after the Cold War. 

	If this technology is developed, then it would be dispersed to all nations cooperating with the treaty. No nation should have to rely on old faulty missile defense programs that have the same success rate as flipping a coin hoping for "heads." This time, nearly two million lives were lost to the decision from a rogue nation, but in the future as weapons continue to be tested with more effective decoy technology, this number is bound to rise, and without a viable means of stopping it, millions of lives are put at risk every passing day. 


\newpage
\begin{thebibliography}{1}
	\bibitem{1}{Lewis, Jeffrey. 2018. \textit{The 2020 Commission Report on the North Korean Nuclear Attacks Against the United States}. Houghton Mifflin Harcourt.}
\end{thebibliography}

\end{document}