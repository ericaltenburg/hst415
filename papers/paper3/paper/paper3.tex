% Turabian Formatting for Research Papers Template, 2018/08/06
%
% Developed using the turabian-formatting package (2018/08/01), available through CTAN: http://www.ctan.org/pkg/turabian-formatting
%
% Additional document class formatting options:
%
% raggedright: ragged right formatting without hyphenations
% authordate: support for the author-date citation style
% endnotes: support for endnotes



\documentclass[12pt]{turabian-researchpaper}


\usepackage[utf8]{inputenc}
\usepackage{csquotes, ellipsis,xurl}

% Specify paper size with geometry package
\usepackage[pass, letterpaper]{geometry}

% For citations, use the biblatex-chicago package
\usepackage{biblatex-chicago}
\addbibresource{works-cited.bib}

% Information for title page
\title{The Only Winning Move is Not to Play}
% \subtitle{A Template based on Turabian's \emph{A Manual for Writers}, 9th edition}
\author{Eric Altenburg}
\course{HST 415: The Nuclear Era}
\date{\today}

\raggedright


\begin{document}

\maketitle

\newpage

Amidst the latter portion of the Cold War when nuclear tensions between the Soviet Union and the United States were at an all time high, \textit{War Games}, a movie directed by John Badham, was released in 1983 and it showed different perspectives about the various attitudes toward nuclear war from a civilian point-of-view. This film follows the actions of David Lightman and his classmate Jennifer Mack after he accidentally finds a way to access the military base NORAD’s artificial intelligence computer named WOPR (War Operation Plan Response). It was recently given control over launching nuclear attacks after a failed test which showed that human nuclear controllers were unwilling to launch missiles and carry out the necessary protocols. Within WOPR are several games to help it learn strategy and various war planning techniques, one of them being Global Thermonuclear War. David initiates a match and WOPR is unable to distinguish between the game and reality, so while he is playing as the Soviets planning to launch several nuclear warheads at the United States virtually, WOPR is simultaneously sending false missile alerts to NORAD causing massive hysteria at the base. Throughout the lifetime of the game, the characters must figure out a way to stop WOPR from causing World War III, and during this, each of them give light to their own unique perspectives toward nuclear operations and how they should be carried out. Once they are put in a stressful situation, their reactions can be interpreted as the the different fears and anxieties each population present during the given time period felt; that of empathy and care juxtaposed with cold strategy and disdain. 

Of the two populations portrayed in the movie, the first shows the attitudes the civilians felt toward the looming thought of nuclear war. In the beginning of the movie, one of the nuclear controllers is unable to carry out the necessary protocols to launch the nuclear missiles, even though it is just a test. Feeling unable to turn the key, he said, "screw the procedure, I want somebody on the damn phone before I kill 20 million people" (Badham 1983). With the single turn of his wrist, he was the only thing in the way of launching ten nuclear missiles with the capability of killing millions. This responsibility weighs heavily, and while it is true that the officers were just following procedure, robbing twenty million people of their livelihood is not fair. No matter how much training they receive, to take away lives is not something anyone is prepared for and the example with this control officer perfectly exemplifies it. 

David and Jennifer also share similar fears and anxieties that can generally be categorized as having a genuine worry of nuclear war as it will lead to their death. Toward the climax of the movie when World War III seems imminent, the creator of WOPR, Professor Falken, goes on to say that "starting over" is a part of nature; it happened with the dinosaurs, therefore, it is bound to happen to humans as well. To him, he feels that nuclear war is inevitable, and if the event were to come to fruition, then being obliterated by the blast as opposed to surviving the nuclear aftermath is the best option. There are two dynamics being unraveled here, the first of clinging onto life because of a fear of nuclear weapons and the other finally letting go of their life because nuclear war is inevitable. 

The opposite mindset toward nuclear weapons is portrayed through NORAD and its commanding officers. To ordinary people like Professor Falken, David, Jennifer, and the nuclear control officer life is something that is precious, they believe there is no winning in war, only death. However, in the war room the high ranking officials truly believe there is such a thing as winning and losing. WOPR was originally put in use because it took out the human component in nuclear decision making, it eliminated the empathy involved in the process and replaced it with a binary yes or no. The terminology being used is similar to that of the technostrategic talk discussed in Carol Cohn's article "Sex and Death in the Rational World of Defense Intellectuals." There she talks about the idea of the specific language often used to discuss nuclear topics and how they are comprised of layers that make it impossible to inject empathy in a conversation about death at the hands of nuclear weapons. She says, "these abstract systems were developed as a way to make it possible to 'think about the unthinkable'—not as a way to describe or codify relations on the ground" (Cohn 709). By using this language, there is no longer the thought of death in the way that books such as \textit{Hiroshima} by John Hersey describe it. This includes the grueling death and conditions many survivors go through, or the trauma that weighs on them many years into the future. None of this is considered in the war room, the only fear that they have in the moment is whether they will win or lose in that exact moment. 

It is also important to take note of the system this film revolves around—WOPR. This computer was designed to treat Global Thermonuclear Wars as games and see everything in terms of a number. But it should be mentioned how eerily similar the program's approach to war is with that of Herman Kahn. In his book, \textit{On Thermonuclear War}, he discusses various nuclear strategies and how to come out of a thermonuclear war on top. Specifically, he talks about the casualties of war in terms of numbers and how coming out of a battle with twenty million dead as opposed to forty million is advantageous; this notion completely shifts the line of thinking where it is no longer based on empathy for those whose loved ones have died, instead, it is now the benefits of not having as many dead (Kahn 20). This is exactly what it seems WOPR was made for—to turn thermonuclear war into a game of numbers; to see how to possibly win at such a war and how to win with the minimal amount of lives lost regardless of how the civilians of each nation feel. WOPR is getting rid of the feeling the aforementioned nuclear control officer felt when he thought he was robbing millions of their livelihood.

Through the clever writing of the characters in \textit{War Games}, Badham was able to convey several fears and anxieties in two categories; those who were terrified and opposed nuclear war with those who felt like it is purely a numbers game.  Between the fear of death that is common in every nation and the fear of losing a "global thermonuclear war" with the Soviet Union, each side—both militaristic and civilian—have their own set of anxieties. In taking out the human element with launching nuclear attacks by implementing WOPR at the core of the nuclear arsenal, it heightens the fears that were already there to begin with. Before, a nuclear launch could be stopped by an officer not willing to take the responsibility of killing millions, but with WOPR, there is no one to put that responsibility on.

\newpage
\begin{thebibliography}{1}
	\bibitem{1} Badham, John, dir. 1983. \textit{War Games}. Metro-Goldwyn-Mayer. Accessed November 9, 2020.
	\bibitem{2} Carol Cohn, "Sex and Death in the Rational World of Defense Intellectuals," Signs 12, no. 4 (Summer 1987): 687-718.
	\bibitem{3} Herman Kahn, On Thermonuclear War (Princeton University Press, 1961), preface and chapter 1.
	% \bibitem{Three} AtomicArchive. 1949. "General Advisory Committee's Majority and Minority Reports on Building the H-Bomb." October 20, 1949. \url{https://www.atomicarchive.com/resources/documents/hydrogen/gac-report.html}
	% \bibitem{One} Transcript of an Executive Session. "Development of a Super Weapon." Joint Committee on Atomic Energy (9 January 1950).
	% \bibitem{Two} Wellerstein, Alex. "Nuclear Secrecy, Security, and the Tragedy of Oppenheimer." HST 415: The Nuclear Era (class lecture, Stevens Institute of Technology, October 14, 2020).
	% \bibitem{two} Alperovitz, Gar and Sherwin, Martin J. 2020. "Op-Ed: U.S. leaders knew we didn’t have to drop atomic bombs on Japan to win the war. We did it anyway." \textit{Los Angeles Times}, August 5, 2020. 
	% \url{https://latimes.com/opinion/story/2020-08-05/hiroshima-anniversary-japan-atomic-bombs}.
	% \bibitem{five} Davies, Dave. 2020. "'Fallout' Tells The Story Of The Journalist Who Exposed The 'Hiroshima Cover-Up.'" \textit{Georgia Public Broadcasting}, August 19, 2020. \url{https://www.gpb.org/news/2020/08/19/fallout-tells-the-story-of-the-journalist-who-exposed-the-hiroshima-cover}.
	% \bibitem{three} Fedman, David and Karacas, Cary. 2020. "Opinion: 75 Years On, Remember Hiroshima And Nagasaki. But Remember Toyama Too." \textit{NPR}, August 1, 2020. \url{https://www.npr.org/2020/08/01/896627359/opinion-75-years-on-remember-hiroshima-and-nagasaki-but-remember-toyama-too}.
	% \bibitem{four} Hersey, John. 1946. \textit{Hiroshima}. New York: A.A. Knopf.
	% \bibitem{one} Walker, J. Samuel. "Recent Literature on Truman's Atomic Bomb Decision: A Search for Middle Ground." \textit{Diplomatic History} 29, no. 2 (April 2005): 311-334.
	% \bibitem{six} Wellerstein, Alex. "Using the Bomb." HST 415: The Nuclear Era (class lecture, Stevens Institute of Technology, September 15, 2020).
\end{thebibliography}

\end{document}