% Turabian Formatting for Research Papers Template, 2018/08/06
%
% Developed using the turabian-formatting package (2018/08/01), available through CTAN: http://www.ctan.org/pkg/turabian-formatting
%
% Additional document class formatting options:
%
% raggedright: ragged right formatting without hyphenations
% authordate: support for the author-date citation style
% endnotes: support for endnotes



\documentclass[12pt]{turabian-researchpaper}


\usepackage[utf8]{inputenc}
\usepackage{csquotes, ellipsis,xurl}

% Specify paper size with geometry package
\usepackage[pass, letterpaper]{geometry}

% For citations, use the biblatex-chicago package
\usepackage{biblatex-chicago}
\addbibresource{works-cited.bib}

% Information for title page
\title{Educating a New Generation through a Revisionist Lens}
% \subtitle{A Template based on Turabian's \emph{A Manual for Writers}, 9th edition}
\author{Eric Altenburg}
\course{HST 415: The Nuclear Era}
\date{\today}

\raggedright


\begin{document}

\maketitle

Over seventy-five years ago on August 6, 1945, the United States made the decision to use the result of the Manhattan Project—an atomic bomb—on Japan, leaving over one-hundred thousand dead in Hiroshima with another fifty-thousand in Nagasaki. Since then, there have been plenty of debates and various narratives created to help explain the events leading up to the decision to use the atomic bomb, with the two most notable being the traditional and revisionist narratives. The former—more common in those who experienced the war and worked on the Manhattan Project—describes the decision to use the bomb as a careful moral deliberation with the motivation to end the war as soon as possible. The latter cares more about the atomic diplomacy involved with the motivation of scaring the growing Soviet Union into submission, and as a result, those who align with this revisionist narrative tend to feel as though the bomb was not necessary to end the war. These two narratives and everything in-between are explained in more detail within J. Samuel Walker’s “Recent Literature on Truman’s Atomic Bomb Decision: A Search for Middle Ground.” He specifically mentions a “middle-ground” where there exists a narrative not constricted to either a revisionist or traditional role, instead, it has the flexibility of agreeing and disagreeing with certain aspects of the two narratives. This flexibility is what makes it popular with today’s generation as new ideas are constantly being explored leading to more questions being asked. This curiosity coupled with the fact that the majority of those living have not experienced the time period of World War 2 (WWII) leads to the popular narrative being the middle-ground, though it is more aligned with the revisionist arguments due to the education system and various works omitting traditional narrative facts when it comes to the bombing.

The first instance of today’s generation exhibiting a revisionist middle-ground narrative can be observed in the purposeful omission of facts pertaining to the traditional narrative. While describing the history of the bombing on its anniversary, Gar Alperovitz and Martin J. Sherwin speak about the lack of necessity for the atomic bomb in “Op-Ed: U.S. Leaders Knew We Didn’t Have to Drop Atomic Bombs on Japan to Win the War. We Did It Anyway.” They first point out the flaws in the accepted wisdom of WWII in the United States, which mainly follows the traditional narrative of saying the bomb ended the war, and it prevented an invasion with the potential of taking hundreds of thousands of American lives, the authors contest this by stating that Japan would have surrendered even without the use of the bomb, and President Trumand along with his advisors knew this. To back this claim, they reference a diary entry by Truman after he learned of Stalin’s decision to join the war, he wrote, “[Stalin will] be in the Jap War on August 15. Fini Japs when that comes about,” and later said to his wife, “we’ll end the war a year sooner now, and think of the kids who won’t be killed” (Alperovitz and Sherwin 2020). The thought of the Soviet Union being the deciding factor in which the Japanese would surrender was initially made popular by Alperovitz in his book, \textit{Atomic Diplomacy}. Walker talks more on this and its relevance to the revisionist narrative in his article saying, “[the revisionists] maintained that the shock of the Soviet invasion of Japanese-controlled Manchuria might well have been sufficient in itself to force a surrender” (Walker 2005, 313). Not only was the bomb not necessary in the eyes of a revisionist, but they also believed Japan would have surrendered once the Soviets invaded. Besides the overly revisionist narrative present in the article, it is also important to take notice that there is a lack of evidence for the traditional narrative, and with the small amount that there is, it is refuted further showing a revisionistesque narrative presence today.

Another article reflecting on the history marking the anniversary of Hiroshima used other various revisionist narratives in describing the events of WWII. David Fedman and Cary Karacas in “75 Years On, Remember Hiroshima and Nagasaki. But Remember Toyama Too” discuss Japan’s distressed urban population and the lack of trust that began to fester towards the end of the war. Fedman and Karacas explained, “The Japanese state had utterly failed to protect its urban population. As the number of homeless ballooned into the millions, officials feared widespread domestic unrest. A multitude of interlocking factors figured into the calculus of Japan's surrender, and the cumulative effects of these incendiary raids were among them” (Fedman and Karacas 2020). According to this, the atomic bomb did not solely end the war like the traditionalist argument states, rather, a combination of factors led to the surrender. Walker discussed a similar finding from Richard B. Frank, where the railroad system responsible for food distribution in Japan was highly vulnerable to aerial attacks. Because of this, “vast numbers of Japanese faced death from starvation, the emperor’s fear that domestic unrest and internal upheaval posed a greater threat to his status than American forces might have increased to the point where he sought peace” (Walker 2005, 328). Frank was not someone who took sides in either the revisionist or traditional narrative, but instead was trying to find a middle-ground between the two. This is similar to what has been seen thus far, while the two aforementioned articles are not explicitly stated to be in the middle-ground, it is important to note that the full scope of the revisionist narrative is not present. Some key interpretations are omitted for varying reasons, but it can be said that due to new evidence the original motivations and historical interpretations are often refuted due to them being too dogmatic or weak to be accepted unconditionally (Walker 2005, 333). 

As it was mentioned earlier, the vast majority of individuals today have not lived through WWII, and as a result, often turn to second-hand evidence. One example of such evidence is John Hersey’s \textit{Hiroshima} where he follows the lives of six individuals and their lives leading up to and after the detonation of the bomb on August 6, 1945 (Hersey 1946). In an interview between Dave Davies and Lesley M. M. Blume, they discuss the events of Hiroshima and mention the impact Hersey’s book has had on the population. She says, “the world did not know the truth about what nuclear warfare really looks like on the receiving end, or did not really understand the full nature of these then experimental weapons, until John Hersey got into Hiroshima and reported it to the world” (Davies 2020). The stories that Hersey told allowed people to attach survivors’ stories to the statistics given to them. Even today, when it comes to the subject of WWII in schools, simply stating the statistics of the bombs does not paint a full picture and to help fill this void of information, students are often required to read Hersey’s \textit{Hiroshima}. This usually leads to them forming their own narrative of the bomb, and based on the background information presented to them, Hersey will likely lead to a more revisionist-driven narrative with the outcome of the using the bomb being that it was not necessary to end the war at all (Wellerstein 2020). Therefore, with the new generation learning about these events from the perspective of the revisionist narrative, it only makes sense that they would form their own revisionist themed middle-ground narrative.

With the current population of those who fought in WWII along with the workers of the Manhattan Project dwindling, it would follow that the traditional narrative become more unpopular due to it losing its biggest proponents. This allows for more room to be made for those who have a non-binary narrative—as in neither revisionist or traditional—where they can agree with certain interpretations while also discrediting others based on new evidence and findings. The largest precursor to this new wave of narratives not only stems from the declining population of veterans, but also the way in which the new generation is taught about the events of Hiroshima and Nagasaki. Reading articles, books, and listening to interviews where the authors tend to fall on the revisionist side of the argument allows for a majority of the population to become partial to that of what they are being taught; a revisionist’s account. However, it is important to note that although the new narratives are more so aligned with the revisionists, that is not to say they are completely in favor of one over the other. Blind acceptance is no longer common with the old narratives because they are littered with poorly backed claims and weak interpretations. Thus, a level of skepticism must be held at all times along with a desire to be curious and explore what has not been considered.

\newpage

\begin{thebibliography}{6}
	\bibitem{two} Alperovitz, Gar and Sherwin, Martin J. 2020. "Op-Ed: U.S. leaders knew we didn’t have to drop atomic bombs on Japan to win the war. We did it anyway." \textit{Los Angeles Times}, August 5, 2020. 
	\url{https://latimes.com/opinion/story/2020-08-05/hiroshima-anniversary-japan-atomic-bombs}.
	\bibitem{five} Davies, Dave. 2020. "'Fallout' Tells The Story Of The Journalist Who Exposed The 'Hiroshima Cover-Up.'" \textit{Georgia Public Broadcasting}, August 19, 2020. \url{https://www.gpb.org/news/2020/08/19/fallout-tells-the-story-of-the-journalist-who-exposed-the-hiroshima-cover}.
	\bibitem{three} Fedman, David and Karacas, Cary. 2020. "Opinion: 75 Years On, Remember Hiroshima And Nagasaki. But Remember Toyama Too." \textit{NPR}, August 1, 2020. \url{https://www.npr.org/2020/08/01/896627359/opinion-75-years-on-remember-hiroshima-and-nagasaki-but-remember-toyama-too}.
	\bibitem{four} Hersey, John. 1946. \textit{Hiroshima}. New York: A.A. Knopf.
	\bibitem{one} Walker, J. Samuel. "Recent Literature on Truman's Atomic Bomb Decision: A Search for Middle Ground." \textit{Diplomatic History} 29, no. 2 (April 2005): 311-334.
	\bibitem{six} Wellerstein, Alex. "Using the Bomb." HST 415: The Nuclear Era (class lecture, Stevens Institute of Technology, September 15, 2020).
\end{thebibliography}

\end{document}