\documentclass[12pt]{turabian-researchpaper}

\usepackage[utf8]{inputenc}
\usepackage{csquotes,ellipsis,xurl}
\usepackage[pass, letterpaper]{geometry}
\usepackage{biblatex-chicago}

\newcommand\question[2]{\noindent\textbf{#1: #2}}

\title{Week 6}
\subtitle{Response Paper}
\author{Eric Altenburg}
\course{HST 415: The Nuclear Era}
\date{\today}

\raggedright


\begin{document}
	\maketitle

	\question{1}{Questions about this that came to mind during lecture or the readings — at least one good one!}

		\begin{itemize}
			\item In the lecture video, how do you think the idea of secrecy has changed with today's standards in regard to electronic documents and emails? More or less secure than the Manhattan Project days?
			\item For the Rosenberg trial protests, were the people protesting mainly aligned with the Communist party or did they just feel the trial was unjust?
			\item How did you find the original Oppenheimer hearings?
		\end{itemize}

	\question{2}{What does the JCAE hearing above, and the Oppenheimer security clearance hearing, reveal about what the JCAE and AEC were afraid of with regards to personnel security clearances? What kinds of things seem to matter the most to them? What does this tell us about their time?}

		In the JCAE, it seemed as though priorities were dispersed amongst the various members as Senators Connally and Hickenlooper were drawn to different conclusions for the first case involving a woman. The specific case made it a point to show that the woman was a heavy drinker and as a result would be more talkative, she was also noted to be sexually promiscuous. This shows where the JCAE's priorities were at — alcohol and sexual activity. This does not necessarily prove that she is someone who is not loyal and Connally points this out, however, Hickenlooper reminds him that the basis for a personnel security clearance not only involves loyalty, but also character and association.

		The Oppenheimer hearing is not entirely similar in the topics they focus on as they base their decision on the associations category while also noting his loyalty. While reading the Teller testimony at Oppenheimer's hearing, the questions he is asked pertains to the loyalty and whether he has exhibited any behavior to show him being disloyal. Aside from this, the question of his associates was discussed, and this is where I feel as though the time period truly shows in what they were focusing on during the trial. Around this time, there was the Red Scare with McCarthy and the various committees formed in order to find all the Communists in the United States. From the video lecture, it was stated that this whole Oppenheimer hearing snowballed from a letter given to Strauss basically saying that he had associates who associated with Communism and that he was against the H-bomb; all of which were already known. However, due to the current circumstances, his Communist associates seemed to carry a heavier weight than it did previously which ultimately played a part in his trial and subsequent denial of security clearance.

	\question{3}{What does the case of the Rosenbergs and VENONA (discussed in the lecture) tell us about the attitudes towards secrecy and the atomic bomb in the US in the 1950s?}

		The case of the Rosenbergs and VENONA gives light to a drastic change in security with it becoming more consequential. As a result of VENONA, the Soviet spy ring was fleshed out and it showed there were various holes in the way the US was conducting their secrecy, especially with the Manhattan Project and how Fuchs was able to breach it. So upon discovering this, in order to set a tone to prevent something similar in the future, the consequences of violating this secrecy were ramped up significantly, thus, leading to the death penalty for the Rosenbergs. However, because of this increase in consequence there was an equal and opposite backlash seen in the form of protests against death sentence trial. In summary, this new attitude can be seen as having two perspectives with one being the government's consequence increase while the rest were seeing it as more of a power grab and maybe even excessive.


	\question{4}{For the case from the DOE website (a present-day approach) — if you took it as representative of OUR current attitudes towards personnel security clearances, what seems to matter the most? In comparing it with the JCAE and Oppenheimer cases, what does it reveal about the changes that have happened from the 1940s-1950s and the present-day?}

		The case I chose from the DOE website was from April 14, 2016 and it results with an individual not having their security clearances restored. They dive into various topics that are nothing like what was discussed in both the JCAE and Oppenheimer hearings. Like it was mentioned in lecture, there was a short section that investigated the individual's debts and financial information, here, it was cited that in the past, they had failed to state the debts they had when being questioned. This is much different than what was investigated for Oppenheimer or even the first case in the JCAE hearing where they discussed topics such as sexual activity and the amount of people they were associated with who had ties in Communism. In the end, the DOE case stated that this individual's security clearance was not eligible for reinstating because they did not present any evidence that would show they will not endanger the common defense and security and will be clearly consistent with the national interest.


	\question{5}{Lastly — with regards to Oppenheimer, what is your ultimate verdict? If you had been on the board making the judgment about Oppenheimer, how would you have ruled?}

		I am not sure I agree with the idea of Communism association being a driving factor for denying Oppenheimer of his security clearance, but I do believe his character was flawed enough to nearly form a convincing case. As it was mentioned in the lecture, he would often sell out his friends in order to gain "street credit" which is a slimy thing to do. I would say this alone shows such a major flaw in his character that it makes it difficult for someone to really trust him or get close to him. This would in turn mean he does not meet one of the three requirements for security clearance.

		On the other hand however, he was highly respected in the scientific community, and looking back on the Manhattan project, he was quintessential according to Groves as he was able to influence the scientists. This vital role he played in the project surely showed he was extremely loyal. Additionally, had Strauss not received that letter leading to his trial, Oppenheimer's security clearance was set to expire in a few weeks anyway. Because of this alone, I would not choose to revoke his clearance as he contributed such a large amount to the project and as such by not taking it away and letting it be denied "naturally" through expiring, I feel as though it would have been a nice way of "saving face" for him.


\vspace*{\fill}
\noindent\textit{I pledge my honor that I have abided by the Stevens Honor System.} -Eric Altenburg

\end{document}