\documentclass[12pt]{turabian-researchpaper}

\usepackage[utf8]{inputenc}
\usepackage{csquotes,ellipsis,xurl}
\usepackage[pass, letterpaper]{geometry}
\usepackage{biblatex-chicago}

\newcommand\question[2]{\textbf{#1: #2}}

\title{Week 5}
\subtitle{Response Paper}
\author{Eric Altenburg}
\course{HST 415: The Nuclear Era}
\date{\today}

\raggedright

\begin{document}
	\maketitle

	\question{1}{Prepare at least one good question that comes to mind from the reading or lectures. Be prepared to ask it in class if it comes to it! :-)}

		\begin{itemize}
			\item It was mentioned that the Jewish community was heavily influenced by communism as it aligned with their beliefs and issues, did this dissipate once the Cold War started?
			\item Why was the US more focused on technology based surveillance while the Soviets were geared more towards the human-style of surveillance?
			\item How on earth does Lilienthal keep such a detailed journal at specific times? Was this really that common in the time period?
			\item In the JCAE, what did Dr. Oppenheimer mean by saying, "since the Commission's program had been pointed toward this eventuality, there shouldn't be any drastic change?"
		\end{itemize}

	\question{2}{Please also respond with what you think about the US documents (Lilienthal journal entries and the JCAE meeting) — what do they tell you about the top policy discussions about nuclear weapons in the US at the time of the Soviet test's detection?}

		Initially reading through the Lilienthal journal entries, it is not immediately clear what is being discussed in terms of policy discussions about nuclear weapons. He keeps a very exact list of events that happened during his day-to-day life, and so the few bits of information that regard nuclear weapons seem to only be centered around his discussion with the president and various officials once he learns of the Soviet test. During these conversations, especially with the president, he brings up excellent points that have good logical behind the decision. Most notably is his wish to show that the power and confidence of President Truman is still high and he is not scared. Aside from this, the only other notable discussion regarding nuclear weapons in his journal entries is the discussion of ending "the miasma of secrecy—holding a secret when there is no secret" (Atomic Energy Years 572). This is interesting because it highlights the fact that a nuclear bomb has been detonated, and this is not something which can be easily hidden from public view. Trying to keep something a secret which is not necessarily a secret to begin with is a valid enough reason to tell the public about the detonation. The manner in which both of these discussions come about make it seem as though they are all caught off-guard and were not expecting the Soviets to have a bomb ready this soon.

		Again, on the Joint Committee on Atomic Energy, many of the Senators seem to express doubt in the findings of the surveillance mission. They often go back and forth with Dr. Oppenheimer by asking what type of evidence do they have, how they collected said evidence, and if the vote was unanimous. This all seems to express the idea that they were caught off-guard by this test, and that they wanted to be 100\% sure all the information is correct before deciding what to do. From then on, they were set in assuming the Soviets had a bomb, and they take cautionary measures stating, "it would certainly not be safe for the country to proceed on any other assumption than that the Russians had the secret of the atomic bomb and were capable of producing the atomic bomb" (JCAE 4). Saying this makes it seem as though they are vastly under prepared and must assume the worst now. Towards the end of the meeting, the topic shifts towards that of nuclear weapon attack prevention.

		Both of these sources all seem to point towards the idea that everyone who is not a scientist is shocked at this test, and that the Soviets were able to build an atomic bomb much faster than anticipated. However, scientists like Dr. Oppenheimer do not seem all too worried or shocked, it is almost as if they give off an energy similar to "I told you so."

	\question{3}{On Kojevnikov and the lecture — please give an informed and nuanced discussion of the question, "did the Soviets just copy the US atomic bomb?" You do not have to answer in a binary yes or no — but if you want to say it is one way or the other (or neither), explain yourself.}

		I do not necessarily think that the Soviets necessarily "stole" the United State's design of the atomic bomb, but at the same time, I would not say they definitely did not. There are heavy United States influences in both the design and creation of the bomb. However, I do not think the US was their sole source of information, and that they were using all of the espionage info as a sort of Lego instruction manual. Instead, I think it is comparable to them just using the material as a reference in times of complete uncertainty, or to check their work/calculations. Much how many students do homework; they do the math problem then check Chegg after the fact to see if they were right. Or even when people write research papers, they look at the available information, and then form their own idea as to how to make the bomb given their own resources. There is no harm in doing so, and the arguments for saying, "oh well the bombs look similar" have no merit to them in my opinion. As it was said in the lecture video, there are only so many ways you can make something different when it comes to the aerodynamic design of the bomb. All-in-all, I do not think the Soviets copied the bomb like a Lego manual, but at the same time, if I were in their position, since the resources were presented to me, why would I not use it?

	\question{4}{You can also put down any other reflections or thoughts that came to mind while listening to the lecture or reading the sources.}

		I find it interesting how easily and willing individuals were to help the Soviets and become spies. Especially considering the main demographic seemed to be those who were Jewish as their communities seemed to have a heavy communist presence. Though, I think this is seen as interesting in this time period because the Soviets were not seen as "bad guys" yet. The USSR was just another country who happened to be the forefront of communism. Looking back now, sure it is strange seeing all these people work for the Soviets, but this is with the knowledge that the Cold War is approaching.

		

\vspace*{\fill}
\noindent\textit{I pledge my honor that I have abided by the Stevens Honor System.} -Eric Altenburg

\end{document}